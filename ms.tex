\documentclass[preprint2]{aastex62}
\usepackage{apjfonts}

%\topmargin 0.6in

\usepackage{multirow}

%\usepackage{deluxetable}
%\usepackage{color}
%\usepackage[usenames, dvipsnames]{color}
%\definecolor{citeRGB}{rgb}{0,0.1,0.7}
%\usepackage[hyperfootnotes=true,naturalnames=true,letterpaper,pdfstartview=FitH,pdfpagemode=UseNone,colorlinks=true,citecolor=citeRGB]{hyperref}

\gdef\HST{\textit{HST}}
\gdef\fluxcgs{\mathrm{erg~s^{-1}~cm^{-2}}}
\gdef\micront{$\mu$m}
\gdef\micronm{\mu\mathrm{m}}
\gdef\aXe{\texttt{aXe}}
\gdef\flux_radius{\textsc{flux\_radius}}
\gdef\epers{\textit{e}$^{-}$ s$^{-1}$}

\gdef\logOH{12 + \log \left(\mathrm{O/H}\right)}
\gdef\SFRuvir{SFR_\mathrm{UV+IR}}
\gdef\peryr{\mathrm{yr}^{-1}}
\gdef\kms{km\,s$^{-1}$}
\gdef\mum{$\mu\mathrm{m}$}
\gdef\24mum{$24\,\mu\mathrm{m}$}
\gdef\arcsec{^{\prime\prime}}
\gdef\UDFj{UDFj-39546284}
\gdef\compareID{UDF-40106456}
\gdef\Lya{\mathrm{Ly}\alpha}
\gdef\Halpha{\mathrm{H}\alpha}
\gdef\Hbeta{\mathrm{H}\beta}

\newcommand\xxx{{\textcolor{red}{\bf xxx}}}
\newcommand\XXX[1]{{\textcolor{red}{xxx #1}}}

\gdef\HAWKI{\mbox{HAWK-I}}


\shortauthors{Brammer et al.}
\shorttitle{$K_s$-band imaging of the Frontier Fields}
%\slugcomment{Draft (\today)}
%\slugcomment{Submitted to ApJSS}

% \watermark{Draft v1.0 (\today)}
% \setwatermarkfontsize{50pt} 

%\citestyle{aa}

\begin{document}


\title{Ultra-deep $K_s$-band Imaging of the \textit{Hubble} Frontier Fields}

% \footnotetext[*]{Based on observations made with the NASA/ESA \textit{Hubble
% Space Telescope}, programs GO-11640, 12177 and 12328, obtained at the
% Space Telescope Science Institute, which is operated by the Association of
% Universities for Research in Astronomy, Inc., under NASA contract NAS
% 5-26555.}

\author{Gabriel B. Brammer}
\affil{Space Telescope Science Institute, 3700 San Martin Dr., Baltimore, MD 21218, USA}

\correspondingauthor{Gabriel B. Brammer}
\email{brammer@stsci.edu}

% \altaffiltext{6}
% {Department of Astronomy, Yale University, New Haven, CT 06520, USA}
% \altaffiltext{4}
% {Department of Physics, University of Wisconsin-Milwaukee, P.O. Box 413,
% Milwaukee, WI 53201, USA}

\begin{abstract}

We present an overview of the ``KIFF'' project, which provides ultra-deep $K_s$-band imaging of all six of the \textit{Hubble} Frontier Fields clusters Abell 2744, MACS-0416, Abell S1063, Abell 370, MACS-0717 and MACS-1149.  All of these fields have recently been observed with large allocations of Directors' Discretionary Time with the \textit{HST} and \textit{Spitzer} telescopes covering $0.4 < \lambda < 1.6$\,\mum\ and 3.6--4.5\,\mum, respectively.  \textit{VLT}/\HAWKI\ integrations of the first four fields reach 5$\sigma$ limiting depths of $K_s\sim26.0$ (AB, point sources) and have excellent image quality (FWHM$\sim0\farcs4$).  Shorter \textit{Keck}/MOSFIRE integrations of the MACS-0717 (MACS-1149) field better observable in the north reach limiting depths $K_s$=25.5 (25.1) with seeing FWHM$\sim$$0\farcs4$ ($0\farcs5$).  In all cases the $K_s$-band mosaics cover the primary cluster and parallel \textit{HST}/ACS+WFC3 fields.  The total area of the $K_s$-band coverage is 490~arcmin$^2$.  The $K_s$-band at 2.2\,\mum\ crucially fills the gap between the reddest HST filter (1.6\,\mum\,$\sim H$~band) and the IRAC 3.6\,\mum\ passband.  While reaching the full depths of the space-based imaging is not currently feasible from the ground, the deep $K_s$-band images provide important constraints on both the redshifts and the stellar population properties of galaxies extending well below the characteristic stellar mass across most of the age of the universe, down to, and including, the redshifts of the targeted galaxy clusters ($z \lesssim 0.5$).  Reduced, aligned mosaics of all six survey fields are provided accompanying this manuscript.  


\end{abstract}

\keywords{galaxies: evolution --- galaxies: high-redshift --- surveys}

%%%%%%%%%%%%%%%%%%%%%%%%%%%%%%%%%%%%%%%%%%%%%%%%%%%%%%%%%%%%%%%%%%
%                                                                %
%   Introduction
%                                                                %
%%%%%%%%%%%%%%%%%%%%%%%%%%%%%%%%%%%%%%%%%%%%%%%%%%%%%%%%%%%%%%%%%%
\section{Introduction}
\label{s:introduction}

\end{document}
